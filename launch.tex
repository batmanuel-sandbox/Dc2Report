\subsubsection{Launching DC2 via the Orchestration Layer}\label{sMw-orc}

In the LSST reference design, the \emph{orchestration layer} is a
component of the DMS that is responsible for preparing and configuring
a pipeline, executing it on a parallel platform, monitoring its
progress, and cleaning up the results after it has finished.  More
specifically, the preparation and execution of a pipeline would
triggered by an event containing a logical name for the pipeline to
run, along with logical identifiers for the data it is to operate on.
The logical pipeline name would be used to ``look up'' the appropriate
policy files in a policy repository.  The policy files would be
retrieved, possibly tweaked, and then deployed onto the target
platforms as necessary.  

In DC2, the orchestration layer was not a component of focus;
nevertheless, we did require the ability to setup and execute the DC2
pipelines in a consistent and automated way.  This functionality was
built into a package called \code{dc2pipe}.  Even though this
package is not intended to be an attempt to meet the general
orchestration layer requirements, it does share in some of the
concepts from the reference design.  

The pipelines are launched via a core script called \code{launchDC2.py}.
It is configured via a top-level policy file which sets some global
parameters such as the working home directory for the execution
(\code{/lsst/DC2root}), the directory representing the policy
repository, and the file containing the list of exposures to process
through the pipelines.  This policy file also sets some
pipeline-specific parameters.  Most important is the list of nodes to
run the pipeline on along with the number of processes to run on each
node.  When the \code{launchDC2.py} script is executed, the operator
passes in this top-level policy file and a name to give to that
particular execution (the \emph{run-ID}).  To prepare for the
execution, this script will: 
\begin{itemize}
\item Ensure that MPI and ssh are properly configured to execute the
execution;
\item Create the run-specific database tables to be used in the
database;  
\item Create a working home for the execution named after the run-ID 
under \code{/lsst/DC2root} for holding input data, output data, and
configuration information; and
\item Cache copies of the pipeline policy files to the working home.
\end{itemize}

Once the preparation is complete, the script executes the pipelines,
each as an MPI application on the configured head node for the
pipeline.  



