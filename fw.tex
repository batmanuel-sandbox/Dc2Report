% Subsection 3.1: fw -*- latex -*-

\subsection{Framework}
\label{FW}

For DC2, we split the LSST framework into two
parts, MWI (Middleware Interface; \Sec{MWI}) and FW.
Each lies in its own namespace (\code{lsst::mwi} and \code{lsst::fw}).

\subsubsection{C++ Classes}

FW contains the framework classes related to processing astronomical images.
The fundamental objects are:
\begin{itemize}
\item Images and Masks
  \begin{itemize}
  \item \code{Image}:
    A 2-dimensional image.
  \item \code{Mask}:
    A set of 2-dimensional bitmasks.
  \item \code{MaskedImage}:
    A container of two \code{Image}s (the data and its variance) and a \code{Mask}.
  \item \code{WCS}:
    A World Coordinate System class.
  \item \code{Exposure}:
    A \code{MaskedImage} and its \code{WCS}.
  \end{itemize}

  \item Convolution Kernels
    \begin{itemize}
      \item Kernels: \code{AnalyticKernel}, \code{FixedKernel}, \code{Kernel}, \code{LinearCombinationKernel}.
      \item Kernel's spatial variability: \code{Chebyshev1Function1}, \code{GaussianFunction1}, \code{Lanczos\-Function1},
	\code{PolynomialFunction1} (and their 2-dimensional analogs).
      \item Free functions to convolve \code{MaskedImage}s with Kernels.
    \end{itemize}

  \item Miscellaneous:
    \begin{itemize}
    \item \code{MaskedImageFormatter}:
      Using \code{boost::persistence}.
    \item\code{DiskImageResourceFITS}:
      Saving image and mask data to FITS files.
    \item \code{MovingObjectPrediction}:
      A prediction of the current position of an object.
    \item \code{DiaSource}:
      The properties of a source detected in a difference image.
    \item \code{Filter}:
      A photometric filter (e.g.~\textit{g}).
    \end{itemize}
\end{itemize}

All of these classes were implemented for DC2, and make use of
the support classes defined in mwi (e.g.~\code{mwi::data::LsstBase},
\code{mwi::utils::Trace}).  They are documented using doxygen (\Sec{build_docs}),
although the level of the documentation is patchy; in particular,
high-level documentation (motivation and examples) tends to be 
missing. The documentation will be expanded in DC3.

\paragraph{Images and Masks}

\subparagraph{Image and Mask}
The classes that contain pixels (\code{Image}, \code{Mask},
\code{MaskedImage}, and \code{Exposure}) use JPL's VisionWorkbench
(VW) to provide the fundamental pixel access; specifically
\code{Image} and \code{Mask} contain a \code{vw::ImageView}.

We have adopted two idioms to access pixels in DC2: registering
a functor (derived from \code{fw::PixelProcessingFunc}) that should be applied
to each pixel in the image, or explicitly iterating over the pixels.

The functor approach is clean and object-oriented, but not very well
suited to algorithms that process sets of pixels (e.g.~searching an
image for connected pixels over threshold).  This approach was used e.g.~in
measuring the centroid of a detected source.

Explicit pixel access in VW uses an iterator rather than a pointer to
access the pixels, although the VW header files state that the
STL-compliant iterator (\code{vw::PixelIterator}) is slow, and should
not be used in time-critical applications.  An alternative,
(\code{vw::ImageView::pixel\_accessor}) may be used but is not a full
STL iterator; we opted to use accessors for DC2.  These proved not to
be rich enough to easily express common operations (e.g.~returning the
value of the pixel just to the right of the accessor), but minor
enhancements to the API would probably resolve these difficulties.

The implementation of subsets of images is not complete; for example
there is no \code{Image} API to get a sub-\code{Image} that shares
pixels with the full \code{Image}.  The code that uses subimages
does not uniformly and consistently allow for coordinate offsets,
although this is likely to be a fruitful source of confusion in
future data challenges.  How to do this consistently and transparently
is not obvious.

\subparagraph{Mask}
LSST's bitplane class uses a dictionary to associate named planes
(e.g.~\code{SATUR}) with bits (e.g.~\code{0x10}); the planes themselves
are implemented as bitmasks within \code{vw::ImageView} as described above.

This representation was chosen to allow us maximum flexibility --- in
particular to allow modules to add planes as needed, and to import
external masks (e.g.~CFHT bad-pixel masks).  During DC2 we realised
that this flexibility came at a cost;  in order to apply operators
to \code{Mask}s the bitplane dictionaries must first be reconciled,
and this expensive step will be needed whenever different modules
add the same planes in varying orders.

For DC3 we intend to preserve the flexibility of named, extensible
bitplanes, but make the dictionary a class variable rather than an instance
variable (\ticket{322});  we expect this change to remove the
need to reconcile \code{Mask}s.

\subparagraph{MaskedImage}
We represent images with associated masks and variances using \code{MaskedImage},
with pixel accessors similar (but not identical) to \code{vw::ImageView::pixel\_accessor}.
These have proved viable, but not especially flexible;  for example, there is
no clean way to read a neighboring pixel value without also moving the mask and
variance iterators.

\subparagraph{\code{WCS}}
LSST's world coordinate system class is currently implemented using Mark Calabretta's 
\code{wcslib}\footnote{http://www.atnf.csiro.au/people/mcalabre/WCS}
which represents the WCS in terms of FITS keywords, stored internally to \code{WCS}
using \code{DataProperty} (\Sec{DataProperty}).  This has proved satisfactory
for DC2, although the CFHT images caused problems with \code{wcslib}'s \code{wcsfix}
function (\ticket{184}) and \code{wcslib} fails to read some apparently legal
FITS headers (\ticket{259}).

\paragraph{Convolution Kernels}

\subparagraph{Kernel}
\code{fw} supports a number of types of kernel derived from a class \code{Kernel}:
\code{AnalyticKernel},
\code{FixedKernel}, \code{LinearCombinationKernel}.  These can be thought
of as small images, and are used to represent difference kernels (\Sec{ImageSubtraction}), and
will in the future probably also be used to represent point spread functions,
PSFs.

An \code{AnalyticKernel} is defined by a function that can be used to evaluate
any pixel within the kernel; a \code{FixedKernel} has the pixel values provided
explicitly, and a \code{LinearCombinationKernel} is a linear combination of
other kernels (e.g.~a set of Gauss-Hermite functions).  In general kernels
are spatially varying, and \code{fw} provides a number of classes to express
this: \code{Chebyshev1Function1}, \code{GaussianFunction1}, \code{LanczosFunction1},
\code{PolynomialFunction1} (and their 2-dimensional analogs).

Once a (possibly spatially varying) Kernel is defined, it may be
applied to a \code{MaskedImage} to generate the convolution of the
image with the specified kernel, taking proper account of the mask and
variance.  In the future we expect to also support convolving single
\code{Image}s.

There are some concerns with the speed of convolution in DC2.  In the
context of difference imaging, this has been partially resolved by
adding a new kernel type, \code{DeltaFunctionKernel}, and providing
a specialisation of the convolution code to exploit the special structure
of a Kernel with only one non-zero pixel; this specialization has
not been deployed in DC2.

\paragraph{Miscellaneous}

\subparagraph{Persistence}
See the discussion in \Sec{persistence}.  The \code{fw} package
provides support for persistence in terms of classes that
implement \code{boost::persistence} for e.g.~\code{MaskedImage},
and that are able to save image and mask data to FITS files.

\subparagraph{MovingObjectPrediction}
This is a prediction of the current position of an object, including
errors.

\subparagraph{DiaSource}
This holds the properties of a source detected in a difference image.  The
class itself contains a large number of parameters as
expected by the database, but only a few are actually measured
for DC2;  even for these, only place-holder algorithms are
employed.

\subparagraph{Filter}
This identifies a photometric filter (e.g.~\textit{g}).

% \subsubsection{Python}  Moved to build.tex and generalized for all
%  of DC2


%  \subsubsection{Testing}  Moved to build.tex and generalized for all
%  of DC2

\subsubsection{Timings}
\label{fwTiming}

The DC2 implementations of critical functions are still slower than we
need.  A preliminary investigation focussed on the convolution
operations, hoping that lessons learned will apply wherever we look at
pixels.  We are analyzing the code at various levels, including at the
assembler level to understand the efficiency of our most
compute-intensive loops.  It is clear to us that there is work to be
done.  In particular, we find an over-abundance of register loads
(this might be better on more modern processors), and pointer updates
are not simple offsets by compile-time constants (e.g.~4).  It seems
likely that a few well chosen changes to the lowest level accessors
could significantly improve performance while remaining transparent to
the user-level code.

